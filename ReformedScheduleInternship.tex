\documentclass[12pt,english]{article}
\usepackage{mathptmx}
\renewcommand{\familydefault}{\rmdefault}
\usepackage[T1]{fontenc}
%\usepackage[utf8]{luainputenc}
\usepackage[letterpaper]{geometry}
\geometry{verbose,tmargin=3cm,bmargin=2cm,lmargin=25mm,rmargin=25mm}
\usepackage{color}
\usepackage{babel}
\usepackage{footnote}
%\addto\shorthandsspanish{\spanishdeactivate{~<>}}

\usepackage{array}
\usepackage{longtable}
\usepackage{rotating}
\usepackage{float}
\usepackage{rotfloat}
\usepackage{multirow}
\usepackage{amssymb}
\usepackage{graphicx}
\usepackage{setspace}
\usepackage{nomencl}


%Packages of mindmap

\usepackage[utf8]{inputenc}
%\usepackage{dtklogos}
\usepackage{tikz}
\usetikzlibrary{mindmap,shadows}
% Information boxes
\newcommand*{\info}[4][16.3]{%
  \node [ annotation, #3, scale=0.65, text width = #1em,
          inner sep = 2mm ] at (#2) {%
  \list{$\bullet$}{\topsep=0pt\itemsep=0pt\parsep=0pt
    \parskip=0pt\labelwidth=8pt\leftmargin=8pt
    \itemindent=0pt\labelsep=2pt}%
    #4
  \endlist
  };
}

% Tikz

\definecolor{myblue}{HTML}{020364}
\usetikzlibrary{shapes,arrows,matrix,decorations.pathreplacing,shapes.geometric,positioning}  
% Gantt
\usepackage{pgfgantt}
\usepackage{adjustbox} %Ajustar a la página

% the following is useful when we have the old nomencl.sty package
\providecommand{\printnomenclature}{\printglossary}
\providecommand{\makenomenclature}{\makeglossary}
\makenomenclature
\onehalfspacing
\usepackage{hyperref}
\hypersetup{
    colorlinks = true,
    allcolors = blue,    
}
\begin{document}

\begin{abstract}

Nowadays, Lower Limb Prosthesis (LLP) are changing at a very fast pace, due to technological developments implemented in such devices. In addition, users have new demands about their prosthesis and they require absolute comfort and good performance. Unfortunately, the demand of LLP has risen mostly in third world countries because of the increment of vascular diseases (e.g., Diabetes Mellitus). However, people do not have the enough funds to acquire advanced prosthesis that return the capabilities of walking or jogging in a proper way.

Despite the fact that active prosthesis help people to reduce metabolic cost, those are heavier and more expensive than\emph{ Energy Storage and Return}(ESR\nomenclature{ESR}{Energy Storage and Return prosthesis}) prosthesis devices, produce uncomfortable noises and require more maintenance than passive ones. Moreover, components of the bionic prosthesis (i.e., actuators, battery, gearbox, among others) make the system highly inefficient. As a consequence, a higher quantity of external energy is required to allow the user having enough autonomy for a daily use.

The current work is a Ph.D. thesis, which purpose is manufacturing a novel customizable configuration of transtibial prosthesis. This device will provide the positive work needed for an amputee at the final stance phase through a passive dynamic system, it will take advantage of cellular solids properties for recycling the energetic lost at the initial contact of the gait.

To validate the results, a gait analysis is needed to quantify the work-loop of the ankle joint in users, and new measurement methods are available to obtain real-time data on gait. Thus, the devices we intend to acquire through this call will be helpful for other thesis related with the research group. In other words, not only will these devices help this Ph.D. thesis, but also will support other ones. 
\end{abstract}
\section{Work Plan}
\subsection{Objetive 1}
Through the extraction of experimental data, we will identify the biomechanical parameters and the work-loop slope in ESR prosthesis users and non-amputees, aiming to obtain the ankle quasi-stiffness of both cases.
%Obtener mediante extracción experimental de datos en marcha, la curva del ciclo de trabajo en la articulación tibio-astragalina en pacientes amputados unilaterales para determinar las asimetrías y cuantificar las irregularidades por esta patología
\subsubsection*{Activities}
\begin{enumerate}
\item To filter the most useful data found in the literature, in order to use it as a benchmark parameter.
\item To acquire the Inertial Measurement Units (IMU's) %Adquirir los dispositivos de medición inercial.
\item To select the appropriate population for each gait analysis. %Escoger la población adecuada para el análisis biomecánico de marcha. 
\item To obtain biomechanical parameters for getting different ankle quasi-stiffness slopes of different ESR and able-bodied patients. %Ejecutar el diseño experimental de evaluación de la marcha.
\item To make the biomechanical model of each subject with the purpose of acquiring non-measurable variables and predicting biomechanical behaviour through the forward dynamic technique in specific software like OpenSim$\circledR$.
\item To obtain ankle quasi-stiffness slope by the combination of kinetic and kinematic variables.
\item To obtain energetic loss in the collision at initial contact of the gait.
\item To verify biomechanical models with similar publications in literature. \end{enumerate}
\subsection{Objective 2}
In order to take advantage of the IMU's, two Master thesis will use simultaneously these devices, with the purpose of getting the following measurements:

\begin{description}
\item [M.Sc. Thesis 1: ]To obtain gait kinematic variables in order to measure the balance in able-bodied people at different terrains.
\item [M.Sc. Thesis 2: ]To validate a suggested protocol for proprioceptive stimulation in patients with Parkinson disease. 
\end{description}


\section{Schedule}
\begin{ganttchart}[
x unit=0.8cm,
y unit title=0.7cm,
y unit chart=0.8cm,
vgrid,
time slot format=isodate-yearmonth,
compress calendar,
title/.append style={draw=none, fill=myblue},
title label font=\footnotesize\sffamily\bfseries\color{white},
title label node/.append style={below=-1.6ex},
title left shift=.05,
title right shift=-.05,
title height=1,
bar/.append style={draw=none, fill=green!75},
bar height=.6,
bar label font=\normalsize\color{black!50},
group right shift=0,
group top shift=.6,
group height=.3,
group peaks height=.2,
bar incomplete/.append style={fill=brown},
vgrid={*{2}{dotted},{green,ultra thick},*{11}{dotted}}
]{2018-01}{2019-01}
\gantttitlecalendar{year} \\
\ganttset{progress label text={}, link/.style={black, -to}}
\ganttgroup{Ph.D Thesis}{2018-01}{2018-06}\\ 
\ganttbar[progress=100, name=T1A]{Filtering the most useful data}{2018-01}{2018-01} \\
\ganttbar[progress=0, name=T1A]{Acquiring IMU's and Ethical committee}{2018-01}{2018-02} \\
\ganttbar[progress=0, name=T1A]{Selecting the appropriate population}{2018-01}{2018-03} \\
\ganttbar[progress=0, name=T1A]{Obtaining biomechanical model}{2018-02}{2018-04} \\
\ganttbar[progress=0, name=T1A]{Getting ankle quasi-stiffness}{2018-03}{2018-05} \\
\ganttbar[progress=0, name=T1A]{Obtain energetic loss at collision}{2018-05}{2018-06} \\
\ganttgroup{Master Thesis 1}{2018-06}{2018-12} \\
\ganttbar[progress=0, name=T1A]{Selecting the appropriate population}{2018-02}{2018-07} \\
\ganttbar[progress=0, name=T1A]{Designing the experiment}{2018-02}{2018-05} \\
\ganttbar[progress=0, name=T1A]{Adapting IMU's to the experiment}{2018-03}{2018-05} \\
\ganttbar[progress=0, name=T1A]{Obtaining biomechanical model}{2018-03}{2018-05} \\
\ganttbar[progress=0, name=T1A]{Obtain kinematic variables}{2018-04}{2018-07} \\
\ganttgroup{Master Thesis 2}{2018-06}{2018-12} \\
\ganttbar[progress=0, name=T1A]{Searching pathological population}{2018-08}{2018-09} \\
\ganttbar[progress=0, name=T1A]{Making experimental protocol}{2018-09}{2018-10} \\
\ganttbar[progress=0, name=T1A]{Selecting the appropriate population}{2018-08}{2018-10} \\
\ganttbar[progress=0, name=T1A]{Ethical committee}{2018-08}{2018-10} \\
\ganttbar[progress=0, name=T1A]{Performing experimental protocol}{2018-10}{2018-12} \\
\ganttbar[progress=0, name=T1A]{Obtaining and analyzing results }{2018-10}{2019-01} \\
\ganttset{link/.style={black}}
%\ganttlink[link mid=.4]{pp}{T1A}
%\ganttlink[link mid=.159]{pp}{T2A}
\end{ganttchart}
\section{Budget}

\begin{table}[H]
\caption{Description of the needed equipment}
\begin{centering}
\begin{tabular}{|>{\raggedright}p{3cm}|>{\raggedright}p{4cm}|>{\raggedright}p{2cm}|>{\centering}p{3cm}|>{\centering}p{3cm}|}
\hline 
\multirow{2}{3cm}{\textbf{Description}} & \multirow{2}{4cm}{\textbf{Necessity}} & \multirow{2}{2cm}{\textbf{Quantity}} & \multicolumn{2}{c|}{\textbf{Resources}}\tabularnewline
\cline{4-5} 
 &  &  & In kind & External\tabularnewline
\hline 
Inertial Measurement Unit MTi-G-710 GNSS. & To take real-time measurments in pathological and able-bodied human gait & 3 units &  & \$ 15.000.000.00\tabularnewline
\hline 
{\small{} \textbf{Total}} &  &  &  & \$15.000.000.00\tabularnewline
\hline 
\end{tabular}
\par\end{centering}

\end{table}
\end{document}